\section{Aufgabe 3}
    \subsection{a)}
        $$a_n=(-1)^{n+1}\frac{n+1}{3n+2}$$
        \subsubsection{$n=2k$}
            $$a_{2k}=(-1)^{2k+1}\frac{2k+1}{6k+2}=-\frac{2k+1}{6k+2}=-\frac{k(2+\frac{1}{k})}{k(6+\frac{2}{k})}=-\frac{2+\frac{1}{k}}{6+\frac{2}{k}}\to -\frac{2}{6} \text{ für } k \to \infty$$
            $\Rightarrow$ Häufungspunkt bei $-\frac{1}{3}$

        \subsubsection{$n=2k-1$}
             $$a_{2k-1}=(-1)^{2k-1+1}\frac{2k-1+1}{6k-3+2}=\frac{2k}{6k-1}=\frac{2k}{k(6-\frac{1}{k})}=\frac{2}{6-\frac{1}{k}}\to \frac{2}{6} \text{ für } k \to \infty$$
             $\Rightarrow$ Häufungspunkt bei $\frac{1}{3}$

         $$\Rightarrow \limsup_{n \in \NN}a_n=\frac{1}{3} \text{ und } \liminf_{n\in\NN}a_n = -\frac{1}{3}$$

     \subsection{b)}
        Wir verwenden diesmal eine \say{Brute Force} Methode, welche die Eigenschaften vom Kosinus ausnutzt: \\
        $a_0 = cos(0) = 1$\\
        $a_1 = cos(\frac{\pi}{4})=\frac{\sqrt{2}}{2}$ \\
        $a_2 = cos(\frac{2\pi}{4})=cos(\frac{\pi}{2})=0$ \\
        $a_3 = cos(\frac{3\pi}{4})=-\frac{\sqrt{2}}{2}$ \\
        $a_4 = cos(\frac{4\pi}{4})=cos(\pi) = -1$ \\
        $a_5 = cos(\frac{5\pi}{4})=-\frac{\sqrt{2}}{2}$ \\
        $a_6 = cos(\frac{6\pi}{4})=cos(\frac{3\pi}{2})=0$ \\
        $a_7 = cos(\frac{7\pi}{4})=\frac{\sqrt{2}}{2}$ \\
        $a_8 = cos(\frac{8\pi}{4})=cos(2\pi)=1$

        $\Rightarrow$ Aufgrund der grundlegenden Eigenschaft von Kosinus durch den Einheitskreis repräsentiert werden zu können, sind die Häufungspunkte: $\{-1, -\frac{\sqrt{2}}{2},0,\frac{\sqrt{2}}{2},1\}$
        $$\Rightarrow \limsup_{n\in\NN}a_n=1, \liminf_{n\in\NN}a_n=-1$$
